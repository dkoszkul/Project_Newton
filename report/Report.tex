\documentclass[12pt]{article}

\usepackage[version=3]{mhchem} % Package for chemical equation typesetting
\usepackage{siunitx} % Provides the \SI{}{} and \si{} command for typesetting SI units
\usepackage{graphicx} % Required for the inclusion of images
\usepackage{natbib} % Required to change bibliography style to APA
\usepackage{amsmath} % Required for some math elements 
\usepackage{float}
\setlength\parindent{0pt} % Removes all indentation from paragraphs

\renewcommand{\labelenumi}{\alph{enumi}.} % Make numbering in the enumerate environment by letter rather than number (e.g. section 6)

\title{Newton's method} % Title

\author{Dominik \textsc{Koszkul} \\ Micha\l\ \textsc{Oleszczyk}} % Author name

\usepackage{geometry}

\newgeometry{tmargin=2cm, bmargin=2cm, lmargin=2cm, rmargin=2cm}

\begin{document}

\maketitle % Insert the title, author and date

\section{Introduce}
\subsection{Newton's method}
Newton's method in optimization is a numeric method and is used to find local extrema/roots in a defined, differentiable function \textit{f}. In this method we need to construct a sequence $x_n$ from initial point $x_0$ to $x_*$ such that $f'(x_*)=0$. Last point is a local extremum point which we  are looking for.
\subsubsection{The Newton's method iteration}
Let $x_0$ be a point 

\end{document}